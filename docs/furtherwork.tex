
Future expansion of this design is possible and we believe the results are
encouraging.  After assessing the results, we have identified various aspects
of our current design that can be changed to improve the functionality and
classification results.

For example, we would like to use all the methods presented here in conjunction
to improve the accuracy and the robustness of the system.  One such system,
could run the gold comparison method to get a rough approximation of the layout
of the symbols and then use the proximity method to improve the confidence in
the results.

Feature based classification may see improvement from considering other
transformations of the pixels when calculating statistics, such as the distance
from the centre which is especially useful for symmetric shapes.

An interesting experiment would be to use ``ideal'' gold images rather than
mean gold images. We suspect that the performance of the gold method could be
improved by using carefully selected images that better illustrate the intended
shape of the character. Also, adding more than one ``gold'' symbol for each
class may allow for more variations than the mean of each variation.

Another approach that we would have like to attempt is to limit the region of
the cell which the feature based method analyses.  By only taking statistics on
these smaller sections of the images we could eliminate areas of the image that
are largely noise, such as the edges of the cell.

Given the strength of the patterns and relationships found by the proximity
statistics, the extensibility of the feature based method, and the accuracy of
the gold image classifier, we believe the results show promise in a hybrid
approach of the proposed methods to create an accurate, robust, and extensible
classification system.

