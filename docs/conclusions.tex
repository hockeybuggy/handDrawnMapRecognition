Each of our classification methods each have their own strengths and
weaknesses.  The \textbf{Feature-Based method} is able to recognize arbitrary
features but requires intended representations of each of the maps during
training. \textbf{Proximity method} works extremely well with the set of maps
we chose because they adhere to patterns. However to detect these patterns in a
data set the proximity classifier needs a relatively accurate representation of
the map. This method has the added benift of haveing it's output be the same as
it's input.  The \textbf{Gold Comparison} method is able to classify symbols
fairly well without intended representations of the map. However this method
requires gold images as input and the number of features in the output
generated is dependent on the number of gold images.

We believe that for the best results the methods should be used in conjunction.
For example using the gold comparison method we could generate a decent
representation of the map which could then be cleaned up by the Proximity
method.
