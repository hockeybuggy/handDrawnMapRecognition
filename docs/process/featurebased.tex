The Feature-Based approach used here applies convolution filters to the
cell images and then gathers statistics based on the distribution of black
pixels in each of the $x$ and $y$ dimensions. A convolution filter is matrix
of coefficients of odd size. Below is a convolution filter for a mean blur.

\[ F_{blur} = \left[
\begin{array}{ccc}
1/16 & 2/16 & 1/16 \\
2/16 & 4/16 & 2/16 \\
1/16 & 2/16 & 1/16
\end{array}\right] \]

If we consider a single cell of the grid, a portion of the image with one symbol in it,
as a matrix of luminance values like in the matrix $A$. A new image is created as the
result of the convolution matrix being applied to each pixel.

\begin{figure}[h] \[
\begin{array}{lcr}
\begin{array}{ccccc}
\ddots & \vdots & \vdots & \vdots & \iddots \\
\ldots & 64 & 64 & 32 & \ldots \\
\ldots & 32 & \fbox{32} & 128 & \ldots \\
\ldots & 32 & 16 & 32 & \ldots \\
\iddots & \vdots & \vdots & \vdots & \ddots \\
\end{array} &
\begin{array}{c}
64F_{11} + 64F_{12} + 32F_{13} \\
+ 32F_{21} + 32F_{22} + 128F_{23} \\
+ 32F_{31} + 16F_{32} + 32F_{32} \\
\end{array} &
\begin{array}{ccccc}
\ddots & \vdots & \iddots \\
\ldots &  \fbox{48}  & \ldots \\
\iddots & \vdots & \ddots \\
\end{array}
\end{array} \]
\end{figure}

The filtering process repeats for each pixel until all pixels have been replaced.
The resulting image has the luminance matrix $B$.

\[B = \left [
    \begin{array}{r r r r}
        a_{00} & a_{10} & \cdots & a_{n0} \\
        a_{01} & a_{11} & \cdots & a_{n1}\\
        \vdots  & \vdots  & \ddots & \vdots\\
        a_{0m} & a_{1m} & \cdots & a_{nm}\\
    \end{array}
\right ] \]

We define $x$ and $y$ to be the row and column sum functions.

\begin{equation}
x(i) = \sum_{j}{B_{i,j}} \quad
y(j) = \sum_{i}{B_{i,j}}
\end{equation}

Next we record $\mu_x, \mu_y, \sigma_x, \sigma_y$ as the feature set for this filter.
We can then apply another filter, either on this filtered image, or the original image
and record statistics on these.
