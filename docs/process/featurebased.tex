
The feature-based method is used to extract geometric information about the
shape and distribution of the points which make up the symbol. To extract these
features, the approach used here applies convolution filters to the cell
images, and then gathers statistics based on the distribution of black pixels
in each of the $x$ and $y$ dimensions. A convolution filter is a matrix of
coefficients and is of an odd size, such that there is a single coefficent at
the center of the filter. The matrix is used to transform an image into another
image which may make certain attributes, such as edges, more prominent.  For
example, the convolution filter, $F_{blur}$ is a mean blur. The convolution
filter transforms the image by replacing each pixel with a new filtered pixel.
The new pixel is created by multiplying the surrounding values by those in the
convolution filter and summing the result. This sum is the new value for the
pixel.

\[ F_{blur} = \left[
\begin{array}{ccc}
1/16 & 2/16 & 1/16 \\
2/16 & 4/16 & 2/16 \\
1/16 & 2/16 & 1/16
\end{array}\right]
\]

\begin{figure}[h] \[
\begin{array}{lcr}
\begin{array}{ccccc}
\ddots & \vdots & \vdots & \vdots & \iddots \\
\ldots & 64 & 64 & 32 & \ldots \\
\ldots & 32 & \fbox{32} & 128 & \ldots \\
\ldots & 32 & 16 & 32 & \ldots \\
\iddots & \vdots & \vdots & \vdots & \ddots \\
\end{array} &
\begin{array}{c}
64F_{11} + 64F_{12} + 32F_{13} \\
+ 32F_{21} + 32F_{22} + 128F_{23} \\
+ 32F_{31} + 16F_{32} + 32F_{32} \\
\end{array} &
\begin{array}{ccccc}
\ddots & \vdots & \iddots \\
\ldots &  \fbox{48}  & \ldots \\
\iddots & \vdots & \ddots \\
\end{array}
\end{array} \]
\caption{Applying the convolution filter to the centred pixel area}
\label{figure:convolution}
\end{figure}

A filtered image is created as the result of the convolution matrix being
applied to each pixel. Figure \ref{figure:convolution} shows the filter being
applied to a single pixel. The filtering process repeats for each pixel until
all pixels have been replaced and resulting image has a luminance matrix,
which we will call $A$.

\[A = \left [
    \begin{array}{r r r r}
        a_{00} & a_{10} & \cdots & a_{n0} \\
        a_{01} & a_{11} & \cdots & a_{n1}\\
        \vdots  & \vdots  & \ddots & \vdots\\
        a_{0m} & a_{1m} & \cdots & a_{nm}\\
    \end{array}
\right ] \]

We define $x$ and $y$ to be the row and column sum functions.

\[
x(i) = \sum_{j}{A_{i,j}} \quad
y(j) = \sum_{i}{A_{i,j}}
\]

Next we record the means and standard deviations: $\mu_x, \mu_y, \sigma_x,
\sigma_y$ as the feature set for this filter. Then we apply another
filter, either on this filtered image or, on the original image and record
statistics on these as well.

