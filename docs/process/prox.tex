
We want to classify the symbol $x$, and we do this by first calculating the
probability that $x$ belongs to each class, $P(X)\quad X_i = P(x\!\in\! C_i)$.
The proximity classifier is given the class of each symbol in the
cardinal directions: north, south, east, and west. Using this information,
we restrict the probability calculation to only consider the subspace
where a symbol has the given classes neighbouring it. Such a restriction is
the conditional probability:
\[
P(X) = P(X|\,N\!=\!c_1,\,E\!=\!c_2,\,S\!=\!c_3,\,W\!=\!c_4)
\]
In calculating probabilities, we can only approximate them using relative
frequencies collected from our sample data. For a set of $k$ classes, there are $k^4$ possible combinations of neighbouring
cells and too many of these combinations occur rarely in our dataset to make
confident approximations of the true probabilities. To resolve this, we assume that
the conditional probabilities are independent in each direction. Under this
assumption, the calculation becomes much more reasonable for our small data set
since there are only $4k$ possible combinations. Our resulting calculation becomes:

\begin{equation} \label{eq:prox}
P(X) \approx P(X|\,N\!=\!c_1)P(X|\,S\!=\!c_2)P(X|\,E\!=\!c_3)P(X|\,W\!=\!c_4)
\end{equation}

From the definition of conditional probability we have that for some class $c$ in
the direction $D$:
\[ P(X|\,D\!=\!c) = \frac{P(X \cap D\!=\!c)}{P(D=c)} \]
To calculate each of the conditional probabilities,
we compute a table from the samples which records $P(X\cap D\!=\!c)$ and $P(D\!=\!c)$ for each class and direction.
We use the table to calculate the relative conditional frequencies and, under the assumption
of independence, we approximate $P(X)$ using equation ~\ref{eq:prox}.

Having the probability vector $P(X)$ under the assumption of independence, we create an input file for WEKA.
This file includes the probability vector, the neighbouring classes, and the class of the
symbol being classified for training. WEKA creates a decision tree which is used to reduce the error in the probability
calculation introduced under the assumption of independence.

\begin{figure}[h]
\begin{center}
\begin{tikzpicture}[node distance = 4 cm, auto]
    \node[ccell](top){$C_{i,j+1}$};
    \node[ucell, below of=top](center){$X$~?};
    \node[ccell, left of=center](left){$C_{i-1,j}$};
    \node[ccell, right of=center](right){$C_{i+1,j}$};
    \node[ccell, below of=center](bottom){$C_{i,j-1}$};

    \path[line] (top)   -- node[right] {$P_{south}(X | \, C_{i,j+1})$} (center);

    \path[line] (left)  -- node[above] {$P_{east}(X | \, C_{i-1,j})$} (center);
    \path[line] (right) -- node[above] {$P_{west}(X | \, C_{i+1,j})$} (center);
    \path[line] (bottom) -- node[right] {$P_{north}(X | \, C_{i,j-1})$} (center);
\end{tikzpicture}
\end{center}
\caption{This figure describes the probability of class $X$ neighbouring cells $C_{i,j}$}
\end{figure}


Let $V_{d}$ be a vector of probabilities, where $n$ is the number of classes,
$d$ is a direction.

\begin{equation}
V_{d} = \left(P_{d}{(X_{1} | \, C_{d})}, P_{d}{(X_{2} | \, C_{d}),\; \ldots \; , P_{d}{(X_{n} | \, C_{d}})}\right)
\end{equation}

To determine the probability of a cell belonging in a particular class, we assume independence
so that we can multiply the vectors of all directions together using the element-wise product.

\begin{equation}
C_{centre} = V_{north} \circ V_{east} \circ V_{south} \circ V_{west}
\end{equation}

This new vector $C_{centre}$ contains $n$ probabilities that this cell should be
classified as a particular class. One approach to classifying the unknown symbol
would be to choose the class corresponding to the maximum probability in $C_{centre}$.

An example of this process can be seen with the following situation. Suppose
that the training set has the following characteristics:

\begin{table}[h]
\small
\begin{center}
\begin{tabular}{ l | r r r r r r }
              & $c_{B}$& $c_{D}$& $c_{EDGE}$& $c_{F}$& $c_{G}$& $c_{W}$ \\ 
              \hline
    north     & 0      & 0.1558 & 0.0519    & 0.0519 & 0.7403 & 0.0325\\
    east      & 0.0779 & 0.0974 & 0         & 0.1493 & 0.6428 & 0\\
    south     & 0      & 0.2403 & 0.0130    & 0.0065 & 0.7403 & 0\\
    west      & 0.0519 & 0.2273 & 0         & 0.0519 & 0.6428 & 0.0260\\
\end{tabular}
\caption{The probability of the class of neighbours in each direction for a
    grass cell from the map that we created entitled PalletTown} 
\end{center}
\end{table}

\begin{figure}[h]
\begin{center}
\begin{tikzpicture}[node distance = 3.5 cm, auto]
    \node[dcell](top){D};
    \node[ucell, below of=top](center){$X$~?};
    \node[fcell, left of=center](left){F};
    \node[gcell, right of=center](right){G};
    \node[gcell, below of=center](bottom){G};

    \path[line] (center)   -- node[right] {$(X|\,N\!=\!c_{D})$} (top);
    \path[line] (center)   -- node[above] {$(X|\,W\!=\!c_{F})$} (left);
    \path[line] (center)   -- node[above] {$(X|\,E\!=\!c_{G})$} (right);
    \path[line] (center)   -- node[right] {$(X|\,S\!=\!c_{G})$} (bottom);
\end{tikzpicture}
\end{center}
\caption{This situation includes a grass to the south and east, a dirt to the
    north and a forest to the west. It is similar to cell 2,19 in pallet town
    (when indexed from zero at the top left)}
\end{figure}
