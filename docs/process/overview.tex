\begin{figure}[h]
\begin{center}
\includegraphics[width=2in]{PokemonPalletTown.png}
\end{center}
\caption{Pallet Town map from Pok\'{e}mon FireRed \cite{firered}}
\label{fig:pokemon}
\end{figure}

\begin{table}
\label{table:symbols}
\caption{Symbol names and the hand drawn forms}
\begin{center}
\begin{tabular}{llll}
Water & \includegraphics[width=.5in]{water.png} &
Grass & \includegraphics[width=.5in]{grass.png} \\
Rock & \includegraphics[width=.5in]{rocks.png} &
Tree & \includegraphics[width=.5in]{tree.png} \\
Dirt & \includegraphics[width=.5in]{dirt.png} &
Sand & \includegraphics[width=.5in]{sand.png} \\
\end{tabular}
\end{center}
\end{table}


For this study we need many different maps with some consistency in style and
subject matter between them. We chose to recreate maps from the popular popular
video game Pok\'{e}mon using the 6 hand-drawn symbols in Table
\ref{table:symbols}.  Choosing a specific source ensures that our data set is
sampled from a well defined distribution.  An example map from the game is
shown in Figure \ref{fig:pokemon}. There is a sample of the symbol names and
the hand drawn forms in Table \ref{table:symbols}.


